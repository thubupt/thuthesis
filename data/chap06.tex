\chapter{总结和展望}
\label{cha:conclusion}

\section{本文总结}
\label{chap6:conclusion}

LDoS攻击作为隐蔽性极强,而且威力很强大的一种DoS攻击,能够对网络造成极大的威胁。但是,在已有的方案中,存在很多检测LDoS攻击的方案,但是,有效的识别LDoS攻击源并限制LDoS攻击对网络的影响的方案还存在很多的可探索的空间。其中,在SDN网络防御LDoS攻击就是很好的一个方案。利用SDN的特性,通过SDN的全局视野和灵活性能够设计出针对LDoS攻击有效的防御方案。目前,SDN中防御LDoS攻击的工作处于还初级阶段。没有深入研究SDN中LDoS防御方案,所以,在SDN中对LDoS防御机制的研究是相当有意义的。

本文通过对SDN的LDoS攻击实现原理进行研究分析之后,提出两个LDoS防御方案:基于带宽保障的方案和基于动态周期性检测的方案。并且把相应的方案部署在SDN网络中,完成LDoS攻击的防御工作。

\noindent 本文取得的成果如下:

(1)分析了已有的LDoS攻击的防御方案,并对这些方案进行了归纳,找到已有方案的不足之处。然后对SDN中的DoS防御方案进行分析,获得了SDN的方案防御LDoS攻击的盲点。

(2)分析了TCP超时重传机制的原理,并且对LDoS攻击模型进行分析。之后对LDoS攻击的种类进行了归纳。最后,提出了分布式LDoS的方案。

(3)通过分析LDoS攻击的原理,提出了两种LDoS防御方案:基于带宽保障的方案和基于动态周期性检测的方案。第一种方案的核心思想是使用Meter规则绑定非TCP流,并限制非TCP流的聚合吞吐量,从而给TCP流保留部分带宽,TCP流就不会进入超时重传机制而降低吞吐量。方案二的使用周期性和平均欧式距离方法来检测LDoS攻击,并在定位LDoS攻击源之后,在攻击源的入口交换机处使用相应的规则对识别的攻击流进行限制,消除LDoS攻击的影响。

(4)在真实的SDN环境下部署了两个方案。通过实验,讨论了两个方案所使用的参数。在单攻击源和分布式的LDoS攻击下,验证了两种方法的有效性,并探讨不同LDoS攻击下,两种方案所需要的系统开销。

(5)对两种LDoS攻击的防御方案的性能进行比较,找到两种方案适合的场景。提出两种方法有效结合的方法。


\section{未来研究展望}
\label{chap6:expection}

在国内外大量关于LDoS防御的调研工作中,发现了LDoS攻击的防御机制还有很多值得探讨的地方,尤其是SDN这种新型的网络,给LDoS攻击的防御工作带来了更多的解决思路。但是,有一些问题未能考虑周全,希望能够在未来的时间里对LDoS攻击的防御工作进行进一步的探索。探索的方向包括以下几个方面:

(1)本文实验的环境均为带外的SDN网络环境,因此,LDoS攻击的影响范围均为数据链路中的TCP流。考虑到SDN的带内组网存在数据链路和控制链路共用的情况,下一步工作为尝试研究LDoS攻击对SDN带内组网的影响和相应的防御方案。

(2)基于动态周期性检测的方案存在不足,在LDoS攻击的$T$过小的时候,防御方案所需要的系统开销会增大很多。在未来的研究中,使用更好的方法来测试LDoS攻击的周期性将会降低该方案的开销,提高方案的可用性。

(3)在网络试验中,本文所使用的网络拓扑相对比较简单,下一步工作,考虑使用更加复杂的SDN网络对防御方案进行测试。

总之,使用SDN的特性帮助解决已有的网络问题是一个很好的方法,尤其是对LDoS攻击的防御工作,更是有极大的实现价值和研究意义。目前的LDoS防御工作还有很多地方需要完善,使用新的网络框架来解决以前难以解决的问题能够给网络安全作出更多的贡献。