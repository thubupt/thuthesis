\begin{resume}

  \resumeitem{个人简历}

  1994 年 8 月 18 日出生于 广西 省 永福 县。

  2012 年 9 月考入 北京邮电 大学 信息与通信工程 学院 信息 专业,2016 年 7 月本科毕业并获得 工学 学士学位。

  2016 年 9 月免试进入 清华 大学 计算机科学与技术 系攻读 硕士 学位至今。

  \researchitem{发表的学术论文} % 发表的和录用的合在一起

  % 1. 已经刊载的学术论文(本人是第一作者,或者导师为第一作者本人是第二作者)
  % \begin{publications}
  %   \item Yang Y, Ren T L, Zhang L T, et al. Miniature microphone with silicon-
  %     based ferroelectric thin films. Integrated Ferroelectrics, 2003,
  %     52:229-235. (SCI 收录, 检索号:758FZ.)
  %   \item 杨轶, 张宁欣, 任天令, 等. 硅基铁电微声学器件中薄膜残余应力的研究. 中国机
  %     械工程, 2005, 16(14):1289-1291. (EI 收录, 检索号:0534931 2907.)
  %   \item 杨轶, 张宁欣, 任天令, 等. 集成铁电器件中的关键工艺研究. 仪器仪表学报,
  %     2003, 24(S4):192-193. (EI 源刊.)
  % \end{publications}

  % 2. 尚未刊载,但已经接到正式录用函的学术论文(本人为第一作者,或者
  %    导师为第一作者本人是第二作者)。
  \begin{publications}[before=\publicationskip,after=\publicationskip]
    % \item Yang Y, Ren T L, Zhu Y P, et al. PMUTs for handwriting recognition. In
    %   press. (已被 Integrated Ferroelectrics 录用. SCI 源刊.)

    \item Xie R J, Xu M W, Cao J H, Li Q. SoftGuard: Defend against the Low-Rate TCP Attack in SDN. In press. (已被ICC录用)
  \end{publications}

  % 3. 其他学术论文。可列出除上述两种情况以外的其他学术论文,但必须是
  %    已经刊载或者收到正式录用函的论文。
  \begin{publications}
    \item Cao J H, Li Q, Xie R J, et al. The CrossPath Attack: Disrupting the SDN Control Channel via Shared Links. In press. (已被Usenix Security 录用)
  \end{publications}

  % \researchitem{研究成果} % 有就写,没有就删除
  % \begin{achievements}
  %   \item 任天令, 杨轶, 朱一平, 等. 硅基铁电微声学传感器畴极化区域控制和电极连接的
  %     方法: 中国, CN1602118A. (中国专利公开号)
  %   \item Ren T L, Yang Y, Zhu Y P, et al. Piezoelectric micro acoustic sensor
  %     based on ferroelectric materials: USA, No.11/215, 102. (美国发明专利申请号)
  % \end{achievements}

\end{resume}
