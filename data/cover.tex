\thusetup{
  %******************************
  % 注意:
  %   1. 配置里面不要出现空行
  %   2. 不需要的配置信息可以删除
  %******************************
  %
  %=====
  % 秘级
  %=====
  secretlevel={秘密},
  secretyear={10},
  %
  %=========
  % 中文信息
  %=========
  ctitle={SDN中LDoS防御机制研究},
  cdegree={工程硕士},
  cdepartment={计算机科学与技术系},
  cmajor={计算机技术},
  cauthor={谢仁杰},
  csupervisor={徐明伟教授},
  % cassosupervisor={陈文光教授}, % 副指导老师
  % ccosupervisor={某某某教授}, % 联合指导老师
  % 日期自动使用当前时间,若需指定按如下方式修改:
  % cdate={超新星纪元},
  %
  % 博士后专有部分
  catalognumber     = {分类号},  % 可以留空
  udc               = {UDC},  % 可以留空
  id                = {编号},  % 可以留空: id={},
  cfirstdiscipline  = {计算机科学与技术},  % 流动站(一级学科)名称
  cseconddiscipline = {系统结构},  % 专 业(二级学科)名称
  postdoctordate    = {2009 年 7 月——2011 年 7 月},  % 工作完成日期
  postdocstartdate  = {2009 年 7 月 1 日},  % 研究工作起始时间
  postdocenddate    = {2011 年 7 月 1 日},  % 研究工作期满时间
  %
  %=========
  % 英文信息
  %=========
  etitle={Research on defense of LDoS in SDN},
  % 这块比较复杂,需要分情况讨论:
  % 1. 学术型硕士
  %    edegree:必须为Master of Arts或Master of Science(注意大小写)
  %             “哲学、文学、历史学、法学、教育学、艺术学门类,公共管理学科
  %              填写Master of Arts,其它填写Master of Science”
  %    emajor:“获得一级学科授权的学科填写一级学科名称,其它填写二级学科名称”
  % 2. 专业型硕士
  %    edegree:“填写专业学位英文名称全称”
  %    emajor:“工程硕士填写工程领域,其它专业学位不填写此项”
  % 3. 学术型博士
  %    edegree:Doctor of Philosophy(注意大小写)
  %    emajor:“获得一级学科授权的学科填写一级学科名称,其它填写二级学科名称”
  % 4. 专业型博士
  %    edegree:“填写专业学位英文名称全称”
  %    emajor:不填写此项
  edegree={Master of Engineering},
  emajor={Computer Technology},
  eauthor={Xie Renjie},
  esupervisor={Professor Xu Mingwei},
  % eassosupervisor={Chen Wenguang},
  % 日期自动生成,若需指定按如下方式修改:
  % edate={December, 2005}
  %
  % 关键词用“英文逗号”分割
  ckeywords={SDN, LDoS, 防御方案},
  ekeywords={SDN, LDoS, defense}
}

% 定义中英文摘要和关键字
\begin{cabstract}
  LDoS(Low-rate Denial of Service)攻击作为一种特殊的DoS(Denial of Service)攻击,对互联网有极大的威胁。由于TCP(Transmission Control Protocol)协议的超时重传机制,LDoS攻击可以通过发送周期性的脉冲流引发TCP吞吐量明显的下降。LDoS攻击具有较低的平均速率,因而该攻击很难被检测和限制。

  近来,SDN(Software-Defined Networking)作为一种有潜力的新型网络范式,为DoS攻击的防御工作提供了新的方法。很多基于SDN的防御系统被提出来防御各种各样的DoS攻击。但是,这些方案都没有考虑LDoS攻击。

  本文提出了两种基于SDN的防御方案来有效的限制LDoS攻击:基于带宽保障的方案和基于动态周期性检测的方案。第一种方案首先在交换机的端口上安装特制的Meter规则。接下来,当有新的流进入网络的时候,控制器判断该流是否为TCP流。在新流不是TCP流的情况下,将该流的流表规则与Meter规则绑定。通过Meter规则限制非TCP流的聚合吞吐量,为TCP流保留带宽。这样,TCP流就不会进入超时重传状态,吞吐量可以保持在正常水平。第二种方案通过在交换机的端口上安装特制的流表规则来检测TCP流的聚合吞吐量是否有显著下降。接下来,它通过端口处统计的聚合吞吐量是否存在周期性来确认LDoS攻击。然后,通过平均欧氏距离方法精准识别LDoS攻击流。最后,控制器通过在入口交换机处安装相应的限制规则来有效限制LDoS攻击。

  本文在Floodlight控制器上实现了两种方案。在真实SDN实验中,本文采用了单攻击源和分布式的LDoS攻击来测试方案的有效性。实验结果证明,两种方案都能够有效地防御LDoS攻击,而且只引入了极低的系统开销。最后,本文分析两种方案的适用场景。第一种方案适用于保护重要TCP流,但不适合于广泛部署。第二种方案可广泛部署在网络中以消除LDoS攻击对系统造成的影响。

\end{cabstract}

% 如果习惯关键字跟在摘要文字后面,可以用直接命令来设置,如下:
% \ckeywords{\TeX, \LaTeX, CJK, 模板, 论文}

\begin{eabstract}
  As a special Denial of Service (DoS) attack, the Low-rate Denial of Service (LDoS) attack is essentially a great threat to the Internet. Due to the Timeout retransmission mechanism, It causes significant throughput degradation of TCP flows by generating periodical pulsing flows. Due to its low rate, the attack is difficult to be detected and throttled.

  Recently, Software-Defined Networking (SDN) has emerged as a promising network paradigm. It provides new methods to defend against DoS attacks. Several SDN-based defense systems have been proposed to deal with various Denial of Service (DoS) attacks. However, they fail to consider the low-rate TCP attack.

  In this paper, Two SDN-based defense schemes are proposed to effectively throttle the LDoS attack. The first scheme is based on bandwidth reservation and the second scheme is based on dynamic periodical detection. The first scheme installs Meter rules in ports of switches. If a non-TCP flow enters the network, it will be associated with a meter rule for rate limiting. Through meter rules, the bandwidth is reserved for TCP flows. Thus, TCP flows avoid retransmitting and their aggregated throughput will remain normal. The second scheme detects the attack by installing crafted flow rules to monitor the degradation of aggregated TCP throughput in ports of switches. It confirms the attack by judging whether there is periodicity for aggregated throughput, and accurately identifies attack flows with Mean Euclidean Distance. Identified attack flows will be effectively throttled by installing mitigation rules in ingress switches.

  Two schemes are implemented in the Floodlight controller. The LDoS attack generated by single and multiple sources is applied to evaluate the schemes. Experiments in a real SDN testbed demonstrate their effectiveness for defending against the LDoS attack. Moreover, the schemes introduce a small overhead. Two schemes are suitable for different scenarios. The first scheme is suitable for protecting important flows. However, the wide deployment of the scheme brings a negative impact on other flows. The wide deployment of the second scheme helps to eliminate the influence of the attack.

\end{eabstract}

% \ekeywords{\TeX, \LaTeX, CJK, template, thesis}
