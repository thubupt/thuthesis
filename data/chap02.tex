\chapter{相关研究综述}
\label{cha:relatedWork}

本章的主要内容包括两个部分,第一部分为介绍已有的LDoS攻击的防御方案,第二部分为介绍SDN中DoS攻击对策防御工作。

\section{LDoS的已有防御方案}
\label{chap2:LDoSwork}
之前已经有不少的LDoS攻击的防御方案被提出,经过实验,很多方案都能够对LDoS攻击进行有效的防御,其中。LDoS攻击的防御已有的防御方案主要有三种分类:基于时频域分析的类型、基于转发队列处理的类型和修改协议的类型。以下是对三种类型的方案的讨论。


\subsection{基于时频域分析的LDoS防御方案}
\label{chap2:TFanalysis}



\subsection{基于队列转发处理的LDoS防御方案}
\label{chap2:queanalysis}

\subsection{修改协议的LDoS解决方案}
\label{chap2:promodify}

\section{SDN中DoS攻击的防御方案}
SDN作为一个新型网络,给DoS攻击的防御工作提供了新的思路,打破了之前的DoS攻击防御的限制,使用SDN的特性,采用新的方案更加有效的对LDoS攻击进行防御。其中包括两类比较典型的DoS防御方案,基于信息收发对称性的分析和动态变化带宽来检测DoS攻击。

\subsection{基于信息收发对称性的方案}
\label{chap2:srbalance}


\subsection{基于动态变化带宽的方案}
\label{chap2:dynamicbandwidth}
