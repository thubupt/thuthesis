\chapter{相关研究综述}
\label{cha:relatedWork}

本章的主要内容包括两个部分,第一部分为介绍已有的LDoS攻击的防御方案,第二部分为介绍SDN中DoS攻击的防御工作。

\section{LDoS的已有防御方案}
\label{chap2:LDoSwork}
之前已经有不少的LDoS攻击的防御方案被提出,经过实验,很多方案都能够对LDoS攻击进行有效地防御,其中。针对LDoS攻击的防御方案主要有三种类型:基于时频域分析、基于转发队列处理和修改协议。以下是对三种方案类型的讨论。


\subsection{基于时频域分析的LDoS防御方案}
\label{chap2:TFanalysis}

Sun\cite{b4}等专家设计了基于动态时间规整(Dynamic Time Warping, DTW)算法的分布式检测方案。该方案提出了分布式LDoS攻击模型,在该模型下对LDoS的攻击流的数据进行规范化,将速率比较低的部分先进行过滤,这样可以对背景流量进行处理。然后,使用DTW算法来精准和有效地识别LDoS攻击,通过提取预定义的LDoS攻击模型特征与过滤后流量的特征对比来判定LDoS攻击流。接下来,提出了分布式系统来有效地区分攻击流和正常流,并对LDoS攻击流进行抑制。不过,依然存在一定的不足:在防御分布式LDoS的模型中,该方案并不能对攻击源进行定位。

Chen\cite{b7}等人提出了协同检测过滤(Collaborative Detection and Filtering, CDF)方法来对LDoS攻击进行防御。该方案对所有的TCP/UDP流进行离散傅里叶变换(Discrete Fourier Transform, DFT)然后对功率谱密度(Power Spectrum Density,PSD)进行分析,LDoS攻击流的功率谱密度的低频部分与正常流不一样。LDoS攻击具有周期性,因此,LDoS攻击流在低频部分功率会相对更高。根据这个特征,该方案就能够对所有的流进行区分。不过,该方案需要的采样频率相对比较高,对计算资源的消耗相对比较高。

吴志军\cite{wuLDoSdetect}等学者提出了基于时间窗统计的LDoS防御方案。其核心思想是在网络流吞吐量下降的时候,利用时间窗去统计出现的突变脉冲数量,再根据数量多少去判断攻击是否出现。该类算法拥有比较好的识别率和较低误报率。但是,该方案只能检测出LDoS攻击流,却无法限制LDoS攻击,同时也存在着采样周期如何设定的问题。

Bhuyan\cite{bhuyan2015low}等学者提出一种基于部分秩相关的检测方案来检测LDoS攻击。由于LDoS攻击是针对一些特殊的协议而使用的,所以,一些受害者的流量与攻击者的流量存在一定的相关性,通过对受害者流量和疑似攻击流做相关性分析就可以找到攻击流。但是,该方案无法找到攻击源的信息,不能消除LDoS攻击的影响。

Chen\cite{b3}等学者提出了一种基于频域分析的方案。该方案主要基于检测到某一服务器的流量的频域特征,能够在极短的时间里检测出LDoS攻击,而且易于部署。但是该方案是针对服务器的流量实现的,因此,该方案在攻击源的目的不同时则难以起效。流级别的LDoS攻击检测方案\cite{chen2007spectral}被提出用于检测LDoS攻击。该方案通过每条流的频谱密度来判断LDoS攻击,拥有很好的识别效果,而且能够精准识别LDoS攻击。但是该方案消耗的计算资源相对较多,不利于大规模地部署。

Shevtekar\cite{shevtekar2005low}等学者提出了基于周期性检测的LDoS防御方案。该方案在边缘路由器上存储必要的流量信息。一旦发现有流量突发长度大于服务器的其他连接的RTT,则判定该流为恶意流。该方案能够有效地限制LDoS攻击,但是只能在部分边缘路由器上部署,不可大规模地部署。

Zhang\cite{b1}等学者提出了流级别的LDoS检测方案。该方案提出新的指标拥塞参与率来协助识别LDoS攻击。通过实验证明,LDoS攻击流与正常流的平均拥塞参与率是有明显的区别的,因此使用该指标可以区分LDoS攻击流。该方案通过流级别的数据识别LDoS攻击,拥有很好的性能,但是,引入的计算开销也相对较大,不可大规模部署。

Ren\cite{ren2007pulsing}等学者提出来在移动网络中防御LDoS攻击的方案。该方案对信道频率和重传数量进行了分析,并得出结论,LDoS攻击存在的时候,移动网络存在高吞吐量和延迟抖动。该方案只能适用于移动网络,不适用于广域网。

Luo\cite{luo2009detecting}等学者提出了LDoS防御方案Vanguard,该方案使用CUMUS算法检测攻击引起的流量异常。对比之前的方案,该方案的性能有了极大的提升,但是该方案并没有锁定攻击源的方法,缺少限制LDoS攻击的方法。

\subsection{基于队列转发处理的LDoS防御方案}
\label{chap2:queanalysis}
Kwok\cite{b22}等学者提出了一种动态队列管理技术,该技术的核心思想是对数据流进行加权,然后对可疑流进行限制。该方案通过安装流表规则(Flow Rule)来累计每个流的转发数据,如果通过流表在一段时间内观察到高速率脉冲,则标定该流为攻击流,并优先丢弃该流的数据包。该策略对LDoS攻击流有很强的限制作用,但是存在很大的不足,对于良性UDP流的误判率很高。

Chang\cite{b8}等学者提出了修改队列公平性的方案。该方案的核心思想是通过丢包速率识别被LDoS攻击影响的TCP流并对其进行保护。该方案在每个路由器上维持一个计数器并统计每个可能的受害者的丢包速率,优先丢弃具备高丢包率的LDoS攻击流,这样就可以保证良性TCP流拥有足够的带宽。

Patel\cite{patel2016throughput}等学者展示了已有主动队列管理(Active Queue Management,AQM)机制防御LDoS攻击的性能比较,并且发现随机早检测算法(Random Early Detection, RED)的性能是最优的。该研究对LDoS攻击下各个算法进行了分析比较,但是,它并没有提出精准识别防御LDoS攻击的方法。


\subsection{修改协议的LDoS解决方案}
\label{chap2:promodify}
Kuzmanovic\cite{Kuzmanovic2006Low}等学者也使用对TCP中RTO参数进行随机化的解决方案。由于LDoS是利用协议同质性来实现的攻击,即TCP协议的RTO有一个默认的初始值,因此,定时对RTO进行随机化可以避免LDoS攻击的影响。不过,该方案只能减轻LDoS对TCP流的影响,并不能完全消除LDoS攻击的影响。而且,该方案也有一定的问题,RTO值若是无法与RTT关联,则RTO值的设定将会与网络状态无关,随机的RTO值过大或者过小都可能使TCP的传输发生问题。


\subsection{特殊的LDoS防御方案}
\label{chap2:special-LDoS-Defense}
针对HTTP协议的LDoS攻击也对网络有一定的威胁,因此,也有不少工作针对该类LDoS攻击做相应的防御工作。

Brynielsson\cite{brynielsson2015detectability}等学者分析了针对HTTP协议的LDoS攻击。该类攻击只需要使用少量的攻击流量,就能够完成攻击。该类攻击利用HTTP1.1的持久连接处理的弱点有效地攻击HTTP服务器。该论文使用频谱分析来研究该类攻击,并通过分析发现该类攻击与普通的LDoS攻击的频谱分析结果类似。当存在LDoS攻击时,低频部分的能量比较高。但该类攻击与普通的LDoS攻击不同,随机化攻击有效而且低频部分的能量更低,因此该类攻击的检测难度比普通的LDoS攻击更高。






\section{SDN中DoS攻击的防御方案}
SDN作为一个新型网络,给DoS攻击的防御工作提供了新的思路,打破了之前的DoS攻击防御的限制,使用SDN的特性,采用新的方案更加有效地对LDoS攻击进行防御。很多方案\cite{b9, b16, b10}都对DoS攻击的防御有极好的效果。也有不少防御方案\cite{b10, b12, b13, b15, b18,wang2015floodguard,zhang2017ftguard}被提出来防御SDN中的消耗控制平面的带宽、计算资源和流表资源的DoS攻击\cite{shin2013avant,cao2017disrupting}。不过,目前的防御方案并不能很好的对LDoS攻击进行有效的防御。


Fayaz\cite{b9}等学者提出了Bohatei,该方案能够灵活地防御多种知名的DDoS攻击,例如SYN Flood攻击, DNS放大攻击和UDP Flood攻击。该方案使用基于标签的转发设计了网络方案,避免了现有的SDN解决方案的缺陷。同时,使用了适应性的策略来灵活处理随时间改变的DDoS混合攻击。

Kang\cite{b16}等学者提出了SPIFFY来有效对抗链路洪泛攻击。该方案做了一次速率变换测试,在提高某个瓶颈链路的带宽的时候,观察经过该链路流的变化。DoS攻击流作为不敏感的流几乎不会变化,而正常的流能够敏锐地观察到链路的变化从而提升传输速率,因此,该方案可以精准的识别DoS攻击流。该方案适用于平均速率比较高的洪泛攻击。

Xu\cite{b10}等学者提出了使用SDN对流检测的方法来判断攻击流。在网络中,作为DDoS攻击源的数据包收发数量差距是很大的,因为DDoS攻击源会大量发送数据,并且大量丢包,因此接收的应答数据包很少。根据这个特性,该方案通过安装流表对每个主机的收发状态进行检测即可找出DDoS攻击源。该方案需要匹配的字段很多才能提高识别率,因此需要消耗极大的计算资源。


