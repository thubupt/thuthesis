\chapter{相关研究综述}
\label{cha:relatedWork}

本章的主要内容包括两个部分,第一部分为介绍已有的LDoS攻击的防御方案,第二部分为介绍SDN中DoS攻击对策防御工作。

\section{LDoS的已有防御方案}
\label{chap2:LDoSwork}
之前已经有不少的LDoS攻击的防御方案被提出,经过实验,很多方案都能够对LDoS攻击进行有效的防御,其中。LDoS攻击的防御已有的防御方案主要有三种分类:基于时频域分析的类型、基于转发队列处理的类型和修改协议的类型。以下是对三种类型的方案的讨论。


\subsection{基于时频域分析的LDoS防御方案}
\label{chap2:TFanalysis}

Sun.H\cite{b4}等专家设计了基于动态时间规整(Dynamic Time Warping, DTW)算法的分布式检测方案。该方案提出了分布式LDoS攻击模型,在该模型下对LDoS的攻击流的数据进行规范化,将速率比较低的部分先进行过滤,这样可以对背景流量进行处理。然后,使用DTW算法来精准和有效地识别LDoS攻击,通过提取预定义的LDoS攻击模型特征与过滤后流量的特征对比来判定LDoS攻击流。接下来,提出了分布式系统来有效的区分攻击流和正常流,并对LDoS攻击流进行抑制。不过,依然存在一定的不足。在防御分布式LDoS的模型中,并不能对攻击源进行定位。

Chen.Y\cite{b7}等人提出了协同检测过滤(collaborative detection and filtering, CDF)方法来对LDoS攻击进行防御。该方案对所有的TCP/UDP流进行离散傅里叶变换(DFT)然后对他们的功率谱密度(Power Spectrum Density , PSD)进行分析,LDoS攻击流的功率谱密度的低频部分与正常流不一样,因为LDoS具有周期性,因此,LDoS攻击流在低频部分功率会相对更高。根据这个特征,该方案就能够对所有的流进行区分。不过,该方案需要的采样频率相对比较高,对计算资源的消耗相对比较高。

吴志军\cite{wuLDoSdetect}等学者提出了基于时间窗统计的LDoS防御方案。其核心思想是在网络流吞吐量下降的时候,利用时间窗去统计出现的突变脉冲数量,再根据数量多少去判断攻击是否出现。该类算法拥有比较好的识别率和误报率。但是,该方案只能检测出LDoS攻击流,却无法限制LDoS攻击,同时也存在着采样周期如何设定的问题。



\subsection{基于队列转发处理的LDoS防御方案}
\label{chap2:queanalysis}
Kwok.Y.K\cite{b22}提出了一种动态队列管理技术,该技术的核心思想就对数据流进行加权,然后对可疑流进行限制。该方案通过安装流表(flow table)来累计每个流的转发数据,如果通过流表在一段时间内观察到有段时间的高速率脉冲,则标定该流为攻击流,并优先丢弃该流的数据包。该策略可以在对LDoS攻击流有很强的限制作用,但是存在很大的不足,对于良性UDP流的误判率很高。

Chang.C.W\cite{b8}提出了修改队列公平性的方案,该方案中,每个路由器维持一个计数器并统计每个可能的受害者的丢包速率。


\subsection{修改协议的LDoS解决方案}
\label{chap2:promodify}

\section{SDN中DoS攻击的防御方案}
SDN作为一个新型网络,给DoS攻击的防御工作提供了新的思路,打破了之前的DoS攻击防御的限制,使用SDN的特性,采用新的方案更加有效的对LDoS攻击进行防御。其中包括两类比较典型的DoS防御方案,基于信息收发对称性的分析和动态变化带宽来检测DoS攻击。

\subsection{基于信息收发对称性的方案}
\label{chap2:srbalance}


\subsection{基于动态变化带宽的方案}
\label{chap2:dynamicbandwidth}
